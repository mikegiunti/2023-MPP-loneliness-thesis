\begin{center}
\textbf{Table A3: Behavioral Descriptives of Survey Participants (Random Effects)} \par \vspace{2ex}
\footnotesize
\newcolumntype{Y}{>{\raggedleft\arraybackslash}X}
\begin{tabularx} {12cm} {@{} l Y Y Y @{}}
\toprule
& \multicolumn{2}{c}{Year}  \\
\cmidrule(l{1em}){2-3} 
 & \textbf{2018} & \textbf{2020} & \textbf{Total} \\
\cmidrule(l{1em}){2-3}\cmidrule(l{1em}){4-5}\cmidrule(l{1em}){6-6}
 & Prop. & Prop. &  \\
\midrule 

\textbf{Cohesion Index} \\
Not Fair, Not Helpful, Not Trustworthy & 0.297 & 0.271 & 179 \\
At least two No & 0.246 & 0.172 & 134 \\
At least two Yes & 0.238 & 0.311 & 182 \\
Fair, Helpful, and Trustworthy & 0.219 & 0.246 & 185 \\
\midrule 

\textbf{Offline Political Participation (12 Months)} \\
Not Participated & 0.783 & 0.900 & 550 \\
Participated & 0.217 & 0.100 & 130 \\
\midrule 

\textbf{Online Political Participation (12 Months)} \\
Not Participated & 0.783 & 0.776 & 494 \\
Participated & 0.217 & 0.224 & 186 \\
\midrule 

\textbf{Census Occupation (STEM or Non-STEM)} \\
Non-STEM Occupation & 0.863 & 0.833 & 557 \\
STEM Occupation & 0.137 & 0.167 & 123 \\
\bottomrule
\end{tabularx}
\par\smallskip\noindent\parbox{12cm}{\raggedright \scriptsize \textbf{Note:} Cell numbers indicate the proportion of participants per category per year, while the \emph{total} column indicates the total number of participants within each category across both years. It is assumed that these variables will mediate the effect of our main regressors on happiness.}
\normalsize
\end{center}
